\documentclass{article}

\usepackage{amsmath}
\usepackage{amssymb}
\usepackage{amsthm}
\usepackage{hyperref}


\begin{document}

\title{Solutions to Ian Grojnowski's Number Fields Part II Course}

\maketitle

Questions are taken from \href{https://dec41.user.srcf.net/notes/II_L/number_fields_eg.pdf}{here}.

\section{}

\subsection{\((1+i)\sqrt{3}\)}
The equation,
\begin{equation}
f(x) = x^{4} + 36
\end{equation}
is satisfied by \((1+i)\sqrt{3}\) since
\begin{equation}
\left[(1+i)\sqrt{3}\right]^{4} = \left[6i\right]^{2} = -36
\end{equation}

\subsection{\(i+\sqrt{3}\)}
The equation,
\begin{equation}
f(x) = x^{4} - 4x^{2} + 16
\end{equation}
is satisfied by \(i+\sqrt{3}\).

\subsection{\(2\cos(2\pi/7)\)}
The equation
\begin{equation}
x^{3} + x^{2} - 2x - 1
\end{equation}
is satisfied by \(2\cos(2\pi/7)\), by utilizing the identities,
\begin{equation}
2\cos(2\pi/7) = e^{2\pi i / 7} + e^{-2\pi i / 7}
\end{equation}
And the identity, setting \(\zeta = e^{2\pi i / 7}\),
\begin{equation}
1+\zeta + \zeta^{2} + \zeta^{3} + \cdots + \zeta^{6} = 0
\end{equation}
We just need to verify that this is irreducible. This can be shown easily by reducing the equation to \(\mathbb{Z}_{2}\left[x\right]\), where the equation becomes,
\begin{equation}
x^{3} + x^{2} + 1
\end{equation}
To check that this is irreducible, its easy to verify that \(x\) does not divide the polynomial. We can also check the one other alternative \((x+1)\) and explicitly verify that it doesn't divide the polynomial. Therefore, the original polynomial is irreducible.

\section{}

\section{}
An element, \(x\), is a unit iff \(N(x) = \pm 1\). But the norm of an element \(x = a + b\sqrt{-d}\), is equal to
\begin{equation}
N(x) = a^{2} + db^{2}
\end{equation}
For \(d>1\) its clear this is only satisfied by \(a=\pm 1, b =0\).

\section{}
In \(\mathbb{Z}\left[\sqrt{3}\right]\),

\begin{equation}
11 = (2\sqrt{3} - 1)(2\sqrt{3} + 1)
\end{equation}

and

\begin{equation}
2 = (\sqrt{3} - 1)(\sqrt{3} + 1)
\end{equation}

Therefore the left side of the equation breaks up as,
\begin{equation}
22 = 2 \cdot 11 = (2\sqrt{3} - 1)(2\sqrt{3} + 1)(\sqrt{3} - 1)(\sqrt{3} + 1)
\end{equation}

But we can also see that,
\begin{equation}
5 + \sqrt{3} = (2\sqrt{3} - 1)(\sqrt{3} + 1)
\end{equation}

and

\begin{equation}
5 - \sqrt{3} = (2\sqrt{3} + 1)(\sqrt{3} - 1)
\end{equation}

and so we see that

\begin{equation}
22 = (5 + \sqrt{3})(5 - \sqrt{3}) = (2\sqrt{3} - 1)(2\sqrt{3} + 1)(\sqrt{3} - 1)(\sqrt{3} + 1)
\end{equation}

so that unique factorization still holds.

\end{document}