\documentclass{article}

\usepackage{amsmath}
\usepackage{amssymb}
\usepackage{amsthm}


\begin{document}

\title{Solutions to Milnor's Galois Theory}

\maketitle

\section{Chapter 1}

\subsection{1-1}
Working in $E=\mathbb{Q}\left[\alpha\right] $ with $ \alpha^{3} - \alpha^{2} + \alpha + 2 = 0 $,
\begin{subequations}
\begin{align}
(\alpha^{2} + \alpha + 1)(\alpha^{2} - \alpha) & = & (\alpha^{3} + \alpha^{2} + \alpha)(\alpha - 1) \\
& = & (2\alpha^{2}-2)(\alpha - 1) \\
& = & 2(\alpha^{2}-1)(\alpha - 1) \\
& = & 2(\alpha^{3} - \alpha^{2} - \alpha + 1) \\
& = & 2(-2\alpha-1) \\
& = & -4\alpha-2
\end{align}
\end{subequations}

Computing the inverse of $(\alpha - 1) $,
\begin{subequations}
\begin{align}
1 & = & (a\alpha^{2} + b\alpha + c)(\alpha - 1) \\
& = & a\alpha^{3} + (b-a)\alpha^{2} + (c-b)\alpha - c \\
& = & b\alpha^{2} + (c-b-a)\alpha + (-c-2a) \\
\end{align}
\end{subequations}
Therefore we conclude $ b = 0 $, $ c - a = 0 $, and $ - c - 2a = 1 $ or equivalently, $ a = c = -\frac{1}{3} $.

Verifying the solution:
\begin{subequations}
\begin{align}
-\frac{1}{3}(\alpha^{2}+1)(\alpha-1) & = & -\frac{1}{3}(\alpha^{3}-\alpha^{2}+\alpha-1) \\
& = & -\frac{1}{3}(-3) \\
& = & 1
\end{align}
\end{subequations}

\subsection{1-2}
\begin{equation}
\left[\mathbb{Q}(\sqrt{2}, \sqrt{3}) : \mathbb{Q}\right] = \left[\mathbb{Q}(\sqrt{2}, \sqrt{3}) : \mathbb{Q}(\sqrt{2})\right] \cdot \left[\mathbb{Q}(\sqrt{2}) : \mathbb{Q} \right]
\end{equation}
It's straightforward to see that $ \left[\mathbb{Q}(\sqrt{2}) : \mathbb{Q} \right] $ = 2. Next we need to determine the degree of $ \left[\mathbb{Q}(\sqrt{2}, \sqrt{3}) : \mathbb{Q}(\sqrt{2})\right] $. We can understand this by noting that $ \mathbb{Q}(\sqrt{2}, \sqrt{3}) $ is isomorphic to $ \mathbb{Q}(\sqrt{2})\left[\alpha\right] $ where $ \alpha $ satisfies $ \alpha^{2} - 3 = 0 $.

So we really just need to determine if there are any solutions to this equation in $ \mathbb{Q}(\sqrt{2}) $. We can check this via,
\begin{subequations}
\begin{align}
\alpha^{2} - 3 & = & (\alpha - (a + b\sqrt{2}))(\alpha - (c + d\sqrt{2})) \\
& = & \alpha^{2} - (a+c) - (b+d)\sqrt{2} + (ac + 2bd + (ad + bc)\sqrt{2})
\end{align}
\end{subequations}
where $a, b, c, d \in \mathbb{Q} $.
From this it's clear that we must have $ a = -c $ and $ b = -d $, and then the equation becomes
\begin{equation}
3 = a^{2} + 2b^{2} + 2ab\sqrt{2}
\end{equation}
This has no solutions since we must have $ ab = 0 $.

Therefore, $ \left[\mathbb{Q}(\sqrt{2}, \sqrt{3}) : \mathbb{Q}(\sqrt{2})\right] = 2 $ and
\begin{equation}
\left[\mathbb{Q}(\sqrt{2}, \sqrt{3}) : \mathbb{Q}\right] = \left[\mathbb{Q}(\sqrt{2}, \sqrt{3}) : \mathbb{Q}(\sqrt{2})\right] \cdot \left[\mathbb{Q}(\sqrt{2}) : \mathbb{Q} \right] = 2 \cdot 2 = 4
\end{equation}

\subsection{1-3}
TODO

\subsection{1-4}
TODO

\section{Chapter 2}

\subsection{2-1}
TODO

\subsection{}

\subsubsection{(a)}
Goal: for a field \(F\) of characteristic \(p\neq 0\), show that if $ f(X) = X^{p} - X - a $ is reducible in $F\left[X\right]$, then it splits into distinct factors in $F\left[X\right]$.

We compute \(f'(X)\),
\begin{equation}
f'(X) = pX^{p-1} - 1 = -1
\end{equation}

But if \(f(X)\) had a repeated factor, say \(g(X)\), then we'd have \(g(X) | f(X)\) and \(g(X) | f'(X)\). But since \(f'(X)=-1\) there cannot be any repeated factors.

\subsubsection{(b)}
Goal: show $ f(X) = X^{p} - X - 1 $ is irreducible in \(\mathbb{Q}\left[X\right]\).

We know any solution to this equation $r=c/d$ must have \(c|-1\) and \(d|1\), therefore the only possible solutions are \(1, -1\). We can check these explicitly:
\begin{equation}
1^{p} - 1 - 1 = -1 \neq 0
\end{equation}
For odd \(p\), we check
\begin{equation}
(-1)^{p} + 1 - 1 = -1 \neq 0
\end{equation}
For \(p=2\), we check
\begin{equation}
(-1)^{2} + 1 - 1 = 1 \neq 0
\end{equation}
Therefore \(X^{p}-X-1\) is irreducible in \(\mathbb{Q}\left[X\right]\).

\subsection{}
Construct the splitting field of \(X^{5}-2\) over \(\mathbb{Q}\).

If we define \(\zeta = e^{\frac{2\pi i}{5}}\), then the roots of this equation are given by:
\begin{equation}
\sqrt[5]{2}, \sqrt[5]{2}\zeta, \sqrt[5]{2}\zeta^{2}, \sqrt[5]{2}\zeta^{3}, \sqrt[5]{2}\zeta^{4},
\end{equation}
From this it's clear that \(\mathbb{Q}\left[\sqrt[5]{2}, \zeta\right]\) is the splitting field.

To determine it's degree, we use,
\begin{equation}
\left[\mathbb{Q}(\sqrt[5]{2}, \zeta) : \mathbb{Q}\right] = \left[\mathbb{Q}(\sqrt[5]{2}, \zeta) : \mathbb{Q}(\zeta) \right] \cdot \left[\mathbb{Q}(\zeta) : \mathbb{Q} \right]
\end{equation}
Starting with \(\left[\mathbb{Q}(\zeta) : \mathbb{Q} \right]\), we observe that there's a linear dependency,
\begin{equation}
1 + \zeta + \zeta^{2} + \zeta^{3} + \zeta^{4} = 0
\end{equation}
So that,
\begin{equation}
\left[\mathbb{Q}(\zeta) : \mathbb{Q} \right] = 4
\end{equation}
Next we look at $ \left[\mathbb{Q}(\sqrt[5]{2}, \zeta) : \mathbb{Q}(\zeta) \right] $. It's clear that this equals \(5\). Therefore, we find overall that
\begin{equation}
\left[\mathbb{Q}(\sqrt[5]{2}, \zeta) : \mathbb{Q}\right] = 5 \cdot 4 = 20
\end{equation}

\subsection{}
Construct the splitting field of \(X^{p^{m}}-1\) over \(\mathbb{F}_{p}\left[X\right]\) and determine its degree over  \(\mathbb{F}_{p}\).

Note that we can rewrite this as
\begin{equation}
X^{p^{m}} - 1 = (X-1)^{p^{m}}
\end{equation}

Therefore, the splitting field is just \(\mathbb{F}_{p}\) and the degree of the extension is 1.

\subsection{}
Let \(f \in F\left[X\right]\), where \(F\) is a field of characteristic 0. Let \(d(X) = gcd(f, f')\). Show that \(g(X) = f(X)d(X)^{-1}\) has the same roots as \(f(X)\), and that these are all simple roots of \(g(X)\).

Let \(f'(X) = d(X)b(X)\). Then calculate,

\begin{subequations}
\begin{align}
g'(X) & = & f'(X)d(X)^{-1} - f(X)d(X)^{-2}d'(X) \\
& = & b(X) - g(X)d(X)^{-1}d'(X)
\end{align}
\end{subequations}

But then we can rewrite this as
\begin{equation}
g(X)d'(X) = d(X)\cdot(b(X) - g'(X))
\end{equation}
XXX

TODO

\subsection{}
Let \(f(X)\) be an irreducible polynomial in \(F\left[X\right]\), where \(F\) has characteristic \(p\). Show that \(f(X) = g(X^{p^{e}})\) where \(g(X)\) is irreducible and separable. Deduce that every root of \(f(X)\) has the same multiplicity \(p^{e}\) in any splitting field.

We can write \(f(X)\) as,
\begin{equation}
f(X) = \sum_{i}a_{i}X^{i}
\end{equation}
Suppose we have a root \(\alpha\). Now we can plug in













\end{document}
