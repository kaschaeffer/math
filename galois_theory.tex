\documentclass{article}

\usepackage{amsmath}
\usepackage{amssymb}
\usepackage{amsthm}


\begin{document}

\title{Notes on Galois Theory}
\maketitle

\section{Roots of Unity and Cyclotomic Polynomials}

The n\textsuperscript{th} roots of unity are defined as the solutions to the equation
\begin{equation}
\zeta^{n} = 1
\end{equation}

There are n solutions to this equation given by,
\begin{equation}
\zeta_{n, k} = e^{\frac{2 \pi i k}{n}}
\end{equation}

However, notice that some of these are less interesting solutions (for example $ 1 $ is always an n\textsuperscript{th} root of unity for all $n$). We define the \textbf{primitive roots of unity} to be those for which,
\begin{subequations}
\begin{align}
\zeta^{k} & \neq 1 \quad \forall k < n \\
\zeta^{n} & = 1
\end{align}
\end{subequations}

\subsection{Group structure}

The roots of unity form a group since if $a$ and $b$ are both roots of unity, then $(ab)^{n} = 1$. In addition, $1$ is trivially a root of unity. Finally, inverses exist since $\frac{1}{a}$ is clearly also a root of unity.

What is the group structure?

TODO xxx

\subsection{Digression: Multiplicative and Additive Groups}

This section is a quick digression for understanding the relationship between additive and multiplicative groups. To start, for prime $ p $, we'll focus on the relationship between the multiplicative group, $ \mathbb{F}_{p}^{\times} $, and the additive group of integers mod $ p - 1 $ which we'll write as $ \mathbb{Z}_{p-1} $

To start we'll write out the simplest nontrivial example of
\begin{equation}
\mathbb{F}_{5}^{x} \cong \mathbb{Z}_{4}
\end{equation}

Being very explicit, it's useful to write out the multiplication table. For $ \mathbb{F}_{5}^{x} $,

\begin{table}
\begin{tabular}{ | c || c | c | c | c |}
\hline
   & \textbf{1} & \textbf{2} & \textbf{3} & \textbf{4} \\ \hline\hline
\textbf{1} & 1 & 2 & 3 & 4 \\ \hline
\textbf{2} & 2 & 4 & 1 & 3 \\ \hline
\textbf{3} & 3 & 1 & 4 & 2 \\ \hline
\textbf{4} & 1 & 3 & 2 & 1 \\ \hline
\end{tabular}
\end{table}

and for $ \mathbb{Z}_{4} $ the addition table,

\begin{table}
\begin{tabular}{ | c || c | c | c | c |}
\hline
   & \textbf{0} & \textbf{1} & \textbf{2} & \textbf{3} \\ \hline\hline
\textbf{0} & 0 & 1 & 2 & 3 \\ \hline
\textbf{1} & 1 & 2 & 3 & 0 \\ \hline
\textbf{2} & 2 & 3 & 0 & 1 \\ \hline
\textbf{3} & 3 & 0 & 1 & 2 \\ \hline
\end{tabular}
\end{table}

Looking at this table there are mulitple isomorphisms. Either
\begin{subequations}
\begin{align}
1 & \leftrightarrow 0 \\
2 & \leftrightarrow 1 \\
3 & \leftrightarrow 3 \\
4 & \leftrightarrow 2
\end{align}
\end{subequations}

or 

\begin{subequations}
\begin{align}
1 & \leftrightarrow 0 \\
2 & \leftrightarrow 3 \\
3 & \leftrightarrow 1 \\
4 & \leftrightarrow 2
\end{align}
\end{subequations}

These correspond respectively to the maps group homomorphisms, $ \mathbb{Z}_{4} \rightarrow \mathbb{F}_{5}^{\times} $
\begin{subequations}
\begin{align}
f_{2}: x & \mapsto 2^{x} \\
f_{3}: x & \mapsto 3^{x}
\end{align}
\end{subequations}

More generally we can look at the map, $ \mathbb{Z}_{p-1} \rightarrow \mathbb{F}_{p}^{\times} $, defined by
\begin{equation}
f_{a} : x \mapsto a^{x}
\end{equation}

and verify that it is actually a group homomorphism. We can quickly check that this does not define a homomorphism if $a=0$, so we'll assume $a\neq 0$ for the rest of the discussion.

First we check that it preserves the group operation,
\begin{subequations}
\begin{align}
f_{a}(n)\cdot f_{a}(m) & = a^{n} \cdot a^{m} = a^{n+m} = f(n+m) \quad \qedsymbol
\end{align}
\end{subequations}

We also need to ensure the image of this map is actually in $ \mathbb{F}_{p}^{\times} $ by checking that
\begin{equation}
2^{n} = 0
\end{equation}
has no solutions in $ \mathbb{F}_{p} $. This is obvious since p is prime (ignoring the edge case of $p=2$).

Now that we've established that this is a well-defined homomorphism, we want to now understand when it's an isomorphism. Since both groups are finite, it's enough to check that the map is injective.

Therefore, we want to understand for what values of $ a $ the following equation has nontrivial solutions
\begin{equation}
a^n = 1 \quad \mod p
\end{equation}
where $ 0 \leq n \leq p - 2 $. We know for $a=1$ this equation is true for all values of $n$. Similarly, it's straightforward to see that for $a=-1$ this equation has a nontrival solution when $n=2$. What about other values?

We use the known fact (see Serre for a proof) that $ \mathbb{F}_{p-1}^{\times} $ is cyclic. Suppose we have a generator for $ \mathbb{F}_{p-1}^{\times} $, x (note: there is not deterministic way to quickly find a generator, but there are known polynomial time probabilistic algorithms). Then it's easy to see that for all choices of
\begin{equation}
a = x^{k} \quad s.t. \quad (k, p - 1) = 1
\end{equation}
the map will be an isomorphism. Therefore there are $\phi(p-1)$ distinct isomorphisms.

Phrasing this all in a different way, each isomorphism corresponds to finding a generator for $ \mathbb{F}_{p-1}^{\times} $.

\subsection{Digression Part 2: Non-prime examples}

If p is not prime, we can still look at the multiplicative group - what group is this isomorphic to?

Worked example: $ \mathbb{F}_{10}^{\times} $

NOTE: this is a bad example since it's simple

General case (taken from wikipedia) is that this is \textbf{not} cyclic using Chinese Remainder Theorem

\subsection{Galois Theory of Finite Fields}
TODO: describe the Galois Theory of finite fields which is relatively simple (and kinda boring)

Digression: when looking at groups, a standard way to get "new" groups is to take the direct product, for example:
\begin{equation}
H = G \times G
\end{equation}
We can ask why we can't do a similar operation on fields. Imagine defining a new field,
\begin{equation}
F = E \times E
\end{equation}
But now we have the problem that \((1, 0)\) and \((0, 1)\) are both elements of $ F $ but
\begin{equation}
(1, 0) \cdot (0, 1) = (0, 0) = 0
\end{equation}
But we know that fields can't have zero divisors so $ F $ isn't a field. So we see that we can't take direct products of fields to get new ones.

We want to look at finite extensions of $ \mathbb{F}_{p} $. Supposed the extension, $ E / \mathbb{F}_{p} $, has degree $ [E : \mathbb{F}_{p}] = f $. Then it's straightforward to see that $ E $ must have size $ p^{f} $ (since $ E $ is an $f$-dimensional vector space over $\mathbb{F}_{p}$).

A key result is that finite fields of the same size are \textbf{isomorphic} (TODO: give proof). Note that this is different from groups, where there may be many groups of the same size (for example, there are two groups of order 4, $ Z_{4} $ and $ Z_{2} \times Z_{2} $). The intuitive reason for this is that groups require less structure so there are more different ways that a fixed set of elements can form a group - however, fields are required to have substantially more structure, so much so that the size of the field fixes the field itself.


Therefore, we find that $ E $ must be isomorphic to $ F_{p^{f}} $. Overall this means that the only extensions of $ F_{p} $ are the $ F_{p^{f}} $.

We can show that $ F_{p^{k}} $ is isomorphic to the splitting field of $ X^{p^{k}} - X $ (TODO: show why this is true). Next we can ask if this extension is Galois over $ F_{p} $.

First, this is a splitting field by definition (TODO explain this more).
Second, we need to show show that the extension is separable. We need to verify that the polynomial has no multiple roots. We can check:
\begin{equation}
f'(x) = p^{k}\cdot X^{p^{k} - 1} - 1 = - 1
\end{equation}
But then it's clear that $f'(x)$ and $f(x)$ cannot have any shared zeros, so there are no multiple roots.

Next, we introduce the Frobenius Automorphism, defined as
\begin{equation}
F(e) = e^{p}
\end{equation}
It's easy to see that the fixed field of the Frobenius Automorphism is exactly $ \mathbb{F}_{p} $. One way to see this is that the fixed points of this map are zeros of $X^{p} - X $, which means there can be at most $ p $ solutions. But we also know that every element of $ \mathbb{F}_{p} $ satisfies this equation.

It's now clear that,
\begin{equation}
id, F, F^{2}, F^{3}, \cdots, F^{k-1}
\end{equation}
are all distinct automorphisms (while $F^{k}$ is simply the identity). But we also know from Galois theory that
\begin{equation}
|Gal(\mathbb{F}_{p^{k}}/\mathbb{F}_{p})| = \left[ \mathbb{F}_{p^{k}} : \mathbb{F}_{p} \right] = k
\end{equation}
Therefore, we see that we have found all automorphisms, that this group is generated by $F$ and that the group is isomorphic to $\mathbb{Z}_{k}$.

\subsection{Digression about bounds on numbers of subgroups}
This section is focused on the question of how many subgroups there are of a group of order $ n $. Via Galois theory this is also a statement about the number of intermediate fields of a field extension.

The naive bound is $ 2^{n} $ which comes from simply looking at all possible subsets of the elements of the group. However, we can do better than this by recognizing that each subgroup has to contain the identity element, giving a bound of $ 2^{n-1} $.

To saturate this bound, we need it to be the case that when we add an element to a subgroup we don't need to add it's inverse additional - i.e. we need that each element is its own inverse. This can be achieved in the group $ \mathbb{Z}_{2} \times \mathbb{Z}_{2} \times \cdots $.

However, another challenges is that if we create a subgroup consisting of $ 1, a, b $ we need it to be the case that no other elements are necessary to complete the subgroup - in particular $ab$ must already be in the subgroup. Obviously we cannot have $ ab = a$ or $ab = b$ since then one of the elements would be the identity. If instead we have $ab = 1$ then it's straightforward to show that:
\begin{subequations}
\begin{align}
ab & = 1 \\
a\cdot ab &= a \\
a^{2}b & = a \\
b & = a
\end{align}
\end{subequations}
which is also a contradiction. Therefore, we see that $2^{n-1}$ is not a tight bound.

\end{document}